% !TeX spellcheck = en_GB 

\section{\label{sec::odesolver}Solving the ODE}

From equation (\ref{eqn::matrixForm}) we get a system of $n_D-1$ ordinary differential equation of order 1, where at most the inhomogeneous part can be non-linear. So we have to solve an initial value problem (IVP) of the form
\begin{flalign*}
	M\dot{u}=Au+v(t)
\end{flalign*}
on the interval $t\in\left[t_0,t_{max}\right]$ with initial value $u_0=u(t_0)$. If $M$ is invertible, the ODE can of course also be written as 
\begin{flalign*}
	\dot{u}=M^{-1}(Au+v(t)).
\end{flalign*}
Inverting a matrix of size $(n_D-1)\times(n_D-1)$ would probably at least take $\mathcal{O}(n_D^3)$ operations\cite{li2009fastsolver}. As this would in our case be of order $10^{12}$, we want to avoid this.

To solve the IVP we implemented four different solvers. All of them are one step solvers, meaning to compute the next time step only the time step before is used. Two of the solvers are explicit and two implicit. The advantage of an explicit solver is normally, that for each time step only one matrix multiplication and one vector addition are needed. Since in our case we can't avoid solving a linear system with $M$ in every step, this will not really help. Also explicit methods are not stable -- especially over a long time \cite{dahmen2006numerik}. This is, because of an expectingly stiff matrix and a condition number, which increases with decreasing step width $h$. Normally baking a cake takes more then one hour. Considering, that we want to make at least one time second, this will result in quite many time steps.

In contrast to that implicit methods are stable -- also over a longer time. The cost of that stability is, that we have to solve a linear system every time step. Considering, that in our case we also had to solve a system in the explicit case, this will not make it worse.

The methods we implemented and their iterations are:
\begin{itemize}
	\item Implicit methods
	\begin{enumerate}[label=\arabic*.)]
		\item Forward Euler
		\begin{flalign*}
			u_{n+1}=u_n+hM^{-1}(Au_n+v(t_n))
		\end{flalign*}
		\item Improved Forward Euler
		\begin{flalign*}
			u_{n+1}=u_n+\frac{h}{2}M^{-1}\left(2Au_n+v(t_n)+v(t_{n+1})+hA\cdot M^{-1}\left(Au_n+v(t_n)\right)\right)
		\end{flalign*}
	\end{enumerate}
	\item Explicit methods
	\begin{enumerate}[resume,label=\arabic*.)]
		\item Backwards Euler
		\begin{flalign*}
			(M-hA)u_{n+1}=Mu_n+hv(t_n)
		\end{flalign*}
		\item Crank-Nicholson
		\begin{flalign*}
			(M-\frac{h}{2}A)u_{n+1}=(M+\frac{h}{2}A)u_n+\frac{h}{2}(v(t_n)+v(t_{n+1}))
		\end{flalign*}
	\end{enumerate}
\end{itemize}
So we have to solve a linear system in each time step for each of these methods. Instead of completely solving the system we want to do some steps of an iterative solver. The most convincing solver we found is the preconditioned conjugate gradient (PCG) method. This method is implemented in MATLAB and can called by \emph{pcg}. As preconditioner we chose modified incomplete Cholesky decomposition, which is also implemented and can be called by
\begin{lstlisting}[language=matlab]
	ichol(Matrix,struct('michol','on'))
\end{lstlisting}
in MATLAB. The paper \cite{tatebe1993multigrid} compares the cholesky preconditioner with multigrid as preconditioner. Here we see, that solving a huge matrix is faster using multigrid as preconditioner. But for our case, where the matrix is large but not huge, doing a few steps with Cholesky as preconditioner takes around the same time.
