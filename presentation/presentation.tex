\documentclass{beamer}

\mode<presentation> {
	\usetheme{Madrid}
}

\usepackage[utf8]{inputenc}
\usepackage{graphicx}
\usepackage{amsmath}
\usepackage{hyperref}

\setbeamertemplate{caption}{\insertcaption\par}
\setbeamerfont{caption}{size=\scriptsize}

\newcommand{\includepic}[3]{
	\parbox{0.65\textwidth}{
	#1}\hspace{0.04\textwidth}
	\parbox{0.25\textwidth}{
	\begin{figure}
		\includegraphics[width=0.25\textwidth]{#2}
		\caption{#3}
	\end{figure}
	}
}
\newcommand\laplace{\mathop{}\!\mathbin\bigtriangleup}
\newcommand*{\boxedcolor}{red}
\renewcommand{\boxed}[1]{\textcolor{\boxedcolor}{%
  \fbox{\normalcolor$\displaystyle#1$}}}
	

\title[TMA4212 - Project]{Nonlinear Schrödinger Equation}
\author[Maximilian, Anders, Sander, Hans]{Maximilian Kieler, Anders Sørby, Sander Switsers, Hans Pieper}
\institute[]{
	Numerical solution of differential equations by difference methods
}

\date{\today}

\begin{document}
	\begin{frame}
		\titlepage
	\end{frame}
	
	\begin{frame}
		\frametitle{Overview}
		\tableofcontents
	\end{frame}

	\section{Introduction}
	\subsection{The Equation}
	
	\begin{frame}
		\frametitle{Nonlinear Schrödinger Equation}
		\begin{block}{}
			\begin{align}
				i u_t + u_{xx} = \lambda |u|^2 u &  \text{\quad on } \left[-\pi, \pi \right] \times \left(0, T\right) \nonumber \\
				u(x,0) = u_0(x) &  \text{\quad on } \left[-\pi, \pi \right] \tag{NLS} \label{eq:sg} \\ 
				u(-\pi,t) = u(\pi,t) &  \text{\quad on } \left(0,T \right) \nonumber \\
				\nonumber
			\end{align}
		\end{block}
	\end{frame}
	
	\subsection{Existence and Uniqueness}
	\begin{frame}
		\frametitle{Existence and Uniqueness}
		\begin{block}{Theorem}
			Let $u_0\in H^1(\left[-\pi, \pi \right],\mathbb{C})$ and the energie of $u_0$, $E(u_0)<\infty$ be finite, then there exists a unique solution of \ref{eq:sg}.
		\end{block}
		\begin{equation*}
			E(u_0):=\int\limits_{\left[-\pi, \pi \right]} \frac{1}{2} |\nabla u_0|^2-\frac{1}{4}\lambda|u_0|^4 dx
		\end{equation*}
	\end{frame}
	
	\begin{frame}
		\frametitle{Existence and Uniqueness}
		\begin{block}{Duhamel's principle}
			\begin{equation*}
				u(t) = e^{it\laplace}u_0 - i \int\limits_0^t e^{i(t-s)\laplace}\left(-\lambda|u(s)|^2u(s)\right)ds
			\end{equation*}
		\end{block}
		\begin{block}{Uniqueness}
			Let $u_1, u_2$ two solutions of \ref{eq:sg} then
			\begin{equation*}
				\|u_1(t)-u_2(t)\|_{L^4} \leq c(\lambda,u_0,N(t))T^\frac{5}{4}\sup\limits_{0\leq s\leq t}\|u_1(s)-u_2(s)\|_{L^4} 
			\end{equation*}
			for an arbitrary but fixed $t$ and $T$ can be chosen such that the right hand side is smaller equal the left hand side.
		\end{block}
	\end{frame}
	
	\section{Numerical Method}
	\subsection{Discretisation schemes}
	
	\begin{frame}
		\frametitle{Discretisation schemes}
		\begin{block}{Original}
			\fontsize{7.7}{8}{
				\begin{align*}
					\frac{i U^{n+1}_j}{k} + \frac{U^{n+1}_{j-1} - 2U^{n+1}_j + U^{n+1}_{j+1}}{2h^2} - \frac{\lambda}{2}\boxed{\left| U^{n+1}_j \right|^2}U^{n+1}_j &= \frac{i U^{n}_j}{k} - \frac{U^{n}_{j-1} - 2U^{n}_j + U^{n}_{j+1}}{2h^2} + \frac{\lambda}{2} \left| U^{n}_j \right|^2U^{n}_j \\
					F(\vec U^{n+1}, \vec U^n) &= 0
				\end{align*}
			}%
		\end{block}
		\begin{equation*}
			U(x_j,t_{n+1}) \approx U(x_j,t_n) + U_t(x_j,t_n)\cdot k + \ldots
		\end{equation*}
		\begin{block}{Modified}
			\fontsize{7.7}{8}{
				\begin{flalign*}
					\frac{i U^{n+1}_j}{k} + \frac{U^{n+1}_{j-1} - 2U^{n+1}_j + U^{n+1}_{j+1}}{2h^2} - \frac{\lambda}{2}\boxed{\left| U^{n}_j \right|^2}U^{n+1}_j &= \frac{i U^{n}_j}{k} - \frac{U^{n}_{j-1} - 2U^{n}_j + U^{n}_{j+1}}{2h^2} + \frac{\lambda}{2} \left| U^{n}_j \right|^2U^{n}_j \\
					A(\vec U^{n}) \vec U^{n+1} &= B(\vec U^n) \vec U^n
				\end{flalign*}
			}%
		\end{block}
	\end{frame}
	
	\subsection{Stability}
    \begin{frame}{Stability}
	    \begin{block}{Stability criterion}
	        \begin{equation*}
	            \frac{4h^4 + 9k^2}{k^2 h^4} \leq  \frac{1}{3} |\lambda|( 1 + \mu k)
    	    \end{equation*}
	    \end{block}
	\end{frame}
	
	\section{Convergence}
	\begin{frame}{Complexity}
	    \begin{block}{Crank-Nicolson}
	        \begin{enumerate}
	            \item in space: order 2
	            \item in time: order 2
	        \end{enumerate}
	    \end{block}
	    \begin{block}{linear Crank-Nicolson}
	        \begin{enumerate}
	            \item in space: order 2
	            \item in time: order 1
	        \end{enumerate}
	    \end{block}
	\end{frame}

	\begin{frame}
		\frametitle{Convergence plots}
		\begin{figure}[htbp]
			\begin{minipage}{0.49\linewidth}
				\centering
				\includegraphics[width=\linewidth]{images/CN_err_space.png}
			\end{minipage}
			\hfill
			\begin{minipage}{0.49\linewidth}
				\centering
				\includegraphics[width=\linewidth]{images/CN_err_time.png}
			\end{minipage}%
		\end{figure}%
		\begin{figure}[htbp]
			\begin{minipage}{0.49\linewidth}
				\centering
				\includegraphics[width=\linewidth]{images/CN_LIN_err_space.png}
				\caption{in space}
			\end{minipage}
			\hfill
			\begin{minipage}{0.49\linewidth}
				\centering
				\includegraphics[width=\linewidth]{images/CN_LIN_err_time.png}
				\caption{in time}
			\end{minipage}
		\end{figure}
	\end{frame}
	
	\section{Numerical Implementation}
	\subsection{Crank-Nicolson}
	
	\begin{frame}
		\frametitle{Crank-Nicolson}
		\begin{block}{}
			\fontsize{7.7}{8}{
				\begin{align*}
					\frac{i U^{n+1}_j}{k} + \frac{U^{n+1}_{j-1} - 2U^{n+1}_j + U^{n+1}_{j+1}}{2h^2} - \frac{\lambda}{2}\boxed{\left| U^{n+1}_j \right|^2}U^{n+1}_j &= \frac{i U^{n}_j}{k} - \frac{U^{n}_{j-1} - 2U^{n}_j + U^{n}_{j+1}}{2h^2} + \frac{\lambda}{2} \left| U^{n}_j \right|^2U^{n}_j \\
					F(\vec U^{n+1}, \vec U^n) &= 0
				\end{align*}
			}%
		\end{block}
		\vspace{7pt}
		Using scipy.optimize.newton\_krylov (Python)
		\begin{itemize}
			\item Memory: Jacobian
			\item Runtime: no easy way to parallelisation
		\end{itemize}
		\vspace{7pt}
		\begin{block}{}
		\begin{center}
			\large Only for small system sizes
		\end{center}
		\end{block}
	\end{frame}
	
	\subsection{Linear Crank-Nicolson}
	
	\begin{frame}
		\frametitle{Linear Crank-Nicolson}
		\begin{block}{}
			\fontsize{7.7}{8}{
				\begin{flalign*}
					\frac{i U^{n+1}_j}{k} + \frac{U^{n+1}_{j-1} - 2U^{n+1}_j + U^{n+1}_{j+1}}{2h^2} - \frac{\lambda}{2}\boxed{\left| U^{n}_j \right|^2}U^{n+1}_j &= \frac{i U^{n}_j}{k} - \frac{U^{n}_{j-1} - 2U^{n}_j + U^{n}_{j+1}}{2h^2} + \frac{\lambda}{2} \left| U^{n}_j \right|^2U^{n}_j \\
					A(\vec U^{n}) \vec U^{n+1} &= B(\vec U^n) \vec U^n
				\end{flalign*}
			}%
		\end{block}
		\vspace{7pt}
		Using Petsc and MPI (C/C++)
		\begin{itemize}
			\item Memory: Sparse Matrix
			\item Runtime: parallel (distributed memory)
		\end{itemize}
		\vspace{7pt}
		\fontsize{10}{8}{
			\url{https://www.mcs.anl.gov/petsc/} \\
			\url{https://www.open-mpi.org/} \\
			\url{http://www.ntnu.edu/studies/courses/TMA4280}
		}
	\end{frame}
	
	\begin{frame}
		\begin{center}
			\Large Questions?
		\end{center}
	\end{frame}

\end{document}
