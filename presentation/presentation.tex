\documentclass{beamer}

\mode<presentation> {
	\usetheme{Madrid}
}

\usepackage[utf8]{inputenc}
\usepackage{graphicx}
\usepackage{amsmath}
\usepackage{hyperref}

\setbeamertemplate{caption}{\insertcaption\par}
\setbeamerfont{caption}{size=\scriptsize}

\newcommand{\includepic}[3]{
	\parbox{0.65\textwidth}{
	#1}\hspace{0.04\textwidth}
	\parbox{0.25\textwidth}{
	\begin{figure}
		\includegraphics[width=0.25\textwidth]{#2}
		\caption{#3}
	\end{figure}
	}
}
\newcommand\laplace{\mathop{}\!\mathbin\bigtriangleup}
\newcommand*{\boxedcolor}{red}
\renewcommand{\boxed}[1]{\textcolor{\boxedcolor}{%
  \fbox{\normalcolor$\displaystyle#1$}}}
	

\title[TMA4220 - Project]{How to bake the perfect cake}
\author[Simon, Hans]{Simon Thomä, Hans Pieper}
\institute[]{
	Numerical Solution of Partial Differential Equations Using Element Methods
}

\date{\today}

\begin{document}
	\begin{frame}
		\titlepage
	\end{frame}
	
	\begin{frame}
		\frametitle{Overview}
		\tableofcontents
	\end{frame}

	\section{Introduction}
	\subsection{The Equation}
	
	\begin{frame}
		\frametitle{Heat Equation}
		\begin{block}{}
			\begin{align}
				Equation
			\end{align}
		\end{block}
	\end{frame}
	
	\subsection{Existence and Uniqueness}
	\begin{frame}
		\frametitle{Existence and Uniqueness}
		\begin{block}{Theorem}
			Let $u_0\in H^1(\left[-\pi, \pi \right],\mathbb{C})$ and the energie of $u_0$, $E(u_0)<\infty$ be finite, then there exists a unique solution of \ref{eq:sg}.
		\end{block}
	\end{frame}
	
	\begin{frame}
		\frametitle{Existence and Uniqueness}
		\begin{block}{Duhamel's principle}
		\end{block}
	\end{frame}
	
	\section{Numerical Method}
	\subsection{Discretisation schemes}
	
	\begin{frame}
		\frametitle{Discretisation schemes}
		\begin{block}{Original}
		\end{block}
	\end{frame}
	
	
	\begin{frame}
		\begin{center}
			\Large Questions?
		\end{center}
	\end{frame}

\end{document}
